\Introduction
Проблема поиска работы в современном мире является одной из насущных проблем, особенно для молодых специалистов. С другой стороны, поиск и отбор претендентов на вакантные должности является дорогостоящим процессом и для самих компаний-нанимателей. Поэтому нет ничего удивительного в том, что многие компании прибегают к помощи рекрутинговых агентств, которые уже занимаются непосредственным подбором персонала.
 
Рекрутинговым агентствам в свою очередь необходима информация о соискателях. И чем больше информации о большем количестве претендентов им доступна, тем больше вероятность найти подходящих работников на соответствующие вакантные места. Для эти целей данные агентства собирают данные и анализируют различные сайты, такие как социальные сети, базы резюме и им подобные, на которых каждый желающий, пройдя процедуру регистрации, может оставить данные о своем образовании, опыте работы, а также пожелания относительного будущего места работы.

Однако, существует некий уровень недоверия к рекрутинговым агентствам, как со стороны работодателей, так и со стороны соискателей.
Главная претензия работодателей к специалистам по подбору персонала связана с неспособностью рекрутеров учитывать индивидуальные особенности конкретного бизнеса. Глубокое знание отдельных сегментов рынка возможно только при узкой специализации агентства. Такие агентства можно назвать отраслевыми организациями, поскольку их область покрытия рынка вакансий ограничена определенной отраслью.
Кандидатам же часто кажется, что специалисты внутри компании поймут их лучше, с большим вниманием отнесутся к их опыту и личным качествам, так как подбирают "под себя". К тому же усложняется сама процедура трудоустройства - сначала надо пройти предварительный отбор в рекрутинговом агентстве, а затем и в самой компании. Для того чтобы исключить влияние этих негативных факторов, процедура промежуточного отбора должна проходить незаметно для соискателя и без каких-либо временных затрат с его стороны.
В моей работе роль кадрового агентства и базы резюме выполняет сайт, на котором сам соискатель заполняет данные о себе, требования к вакансии, а также выбирает ту область, в которой он бы хотел найти работу.
После этого в зависимости от выбранной области резюме отправляется  в соответствующую отраслевую организацию для дальнейшей обработки.

Поскольку отраслевые организации выполняют лишь предварительный отбор претендентов, данный этап можно полностью автоматизировать, взяв за критерии требования работодателя и возможности, а также пожелания соискателя.

Целью работы является разработка и реализация РСОИ, позволяющей работодателям находить подходящих кандидатов на вакантные места, пользуясь услугами отраслевых агентств, а также подбирать вакансии для соискателей. 

Для достижения поставленной цели необходимо решить следующие задачи:

\begin{enumerate}
\item проанализировать предметную область и определить требования к системе;
\item определить типы систем участников и разработать протокол взаимодействия между ними;
\item реализовать логику работы узлов системы;
\item провести тестирование системы и проверить ее работоспособность.
\end{enumerate}
